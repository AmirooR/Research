\documentclass[a4paper,12pt]{article}
\usepackage{amsmath}
\usepackage{amssymb}
\usepackage{color}
\usepackage{mathtools}
\usepackage{tabularx}
\newcommand{\blue}[1]{\textcolor{blue}{#1}}
\newcommand{\red}[1]{\textcolor{red}{#1}}
\begin{document}

\begin{center}
In The Name of God.

The Merciful, The Compassionate.
\vskip 1cm
{\Large\bfseries{Orthogonal Matrices and Gram Schmidt}}

\vskip 0.2cm
\tiny{notes on Gilbert Strang videos, Lecture 17}
\end{center}

\section{Orthonormality}
\begin{itemize}
\item \textbf{Orthonormal basis}: \\
bases $\{q_1, q_2,\ldots,q_n\}$ are orthonormal vectors:\\

\begin{equation*}
\left\{\begin{array}{rl}
q_i^T.q_j = 0 & \text{if} i \neq j \\
q_i^T.q_j = 1 & \text{if} i = j
\end{array}
\right.
\end{equation*}

\item \textbf{Orthonormal matrices}: columns are orthonormal 
$Q = 
\begin{bmatrix}
\mid & \mid &        & \mid \\
q_1  & q_2  & \ldots & q_n \\
\mid & \mid &        & \mid
\end{bmatrix}
$:
\begin{itemize}
\item $Q^T.Q = I$
\item if $Q$ is squared, $Q^T = Q^{-1}$
\item Suppose $Q$ has orthonormal columns, project onto its columns:
\begin{itemize}
\item $P = Q(Q^T Q)^{-1} Q^T = Q Q^T \rightarrow$ folows the properties. $P = I$ if $Q$ is squared

\item $A^T A \hat{x} = A^T b \Rightarrow Q^T Q\hat{x} = Q^T b \rightarrow \hat{x_i} = q_i^Tb$ 
\end{itemize}
\end{itemize}

\end{itemize}

\section{Gram Schmidt}
\begin{itemize}
\item start with independent vectors $\{a,b,\ldots \}$, find orthogonal vectors $\{A,B,\ldots \}$
and orthonormal ones: $\{q_1 = \frac{A}{\lVert A\rVert},\ldots \}$\\

\item $A = a$, $B$ must be orthogonal to $a$. $B=e$ (in projection). $\Longrightarrow B = b-p=b-A^TbA/(A^T A)$

\item $C = A^T c A / (A^T A) - B^T c B/(B^TB), \ldots$

\item $A = Q R: 
\begin{bmatrix}
\mid & \mid &        & \mid \\
a_1  & a_2  & \ldots & a_n \\
\mid & \mid &        & \mid
\end{bmatrix} = 
\begin{bmatrix}
\mid & \mid &        & \mid \\
q_1  & q_2  & \ldots & q_n \\
\mid & \mid &        & \mid
\end{bmatrix}
\begin{bmatrix}
a_1^Tq_1  & \ldots \\
a_1^Tq_2  & \ldots \\
\vdots    & \vdots
\end{bmatrix}$

$a_1^T q_2$ is $0$, $R$ is upper triangular. because later $q$'s are set to be perpendicular to the earlier ones! ( Look Book)
\end{itemize}

\end{document}
