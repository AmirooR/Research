\documentclass[a4paper,12pt]{article}
\usepackage{amsmath}
\usepackage{amssymb}
\usepackage{amsthm}
\usepackage{color}
\usepackage{mathtools}
\usepackage{tabularx}
\newcommand{\blue}[1]{\textcolor{blue}{#1}}
\newcommand{\red}[1]{\textcolor{red}{#1}}
\theoremstyle{definition} \newtheorem{Theorem}{Theorem}
\begin{document}

\begin{center}
In The Name of God.

The Merciful, The Compassionate.
\vskip 1cm
{\Large\bfseries{Eigenvalues, Eigenvectors, Diagonalization, and Powers of $A$}}

\vskip 0.2cm
\tiny{notes on Gilbert Strang videos, Lecture 21,22}
\end{center}

\section{Eigenvalues and eigenvectors}
\begin{itemize}
	\item $Ax = \lambda x$, eigenvectors parallel to $x$
	\begin{itemize}
		\item  If $A$ is singular, then $\lambda = 0$ is an eigenvalue.
		\item What are the $x$'s, and $\lambda$'s of projection matrix $P$? $\longrightarrow \lambda = 0,1$
		\begin{itemize}
			\item Any $x$ in the projection plane : $Px = x$, $\lambda = 1$
			\item Any $x \perp$ to the plane: $Px = 0$, $\lambda = 0$ 
		\end{itemize}
		\item How about permutation matrix? 
		\begin{itemize}
			\item $A = \begin{bmatrix}0 & 1 \\ 1 & 0  \end{bmatrix}$
			\item swap rows!
			\item $x = \begin{bmatrix} 1 \\ 1 \end{bmatrix}$, $\lambda = 1$, $Ax=x$
			\item $x = \begin{bmatrix} -1 \\ 1 \end{bmatrix}$, $\lambda = -1$, $Ax=-x$
		\end{itemize}
		\item FACT: sum of $\lambda$'s $ = a_{11} + a_{22} + \ldots + a_{nn}$
		\item Solve $Ax = \lambda x$: $(A-\lambda I)x= 0$
		\begin{itemize}
			\item $\Rightarrow det(A-\lambda I) = 0$. Find $\lambda$'s first.
			\item For $2\times 2$ matrices $trace$ is the linear coefficient, and $det$ is constant of 
degree 2 equation for eigenvalues.

			\item For degenerate equations (e.g., when $A$ is upper triangular), the eigenvalues, and eigenvectors are repeated.
A double or higher order degeneracy will cause a lost in one(or more) equations. We have infinite number of ways to
choose eigenvectors. The most traditional way (almost in $2^{nd}$ order degeneracy) is to choose the first one to be the
trivial normalized one. and the other vector to be orthogonal to the first one.
		\end{itemize}
	\end{itemize}
\end{itemize}

\section{Diagonalization and Powers of $A$}
\begin{itemize}
	\item \textbf{Diagonalizing a matrix,  $S^{-1} A S = \Lambda$}:
		\begin{itemize}
			\item Suppose we have $n$ linearly independent eigenvectors of $A$. 
			\item Put them in columns of $S$:\\
			\begin{align*}
				AS &= A \begin{bmatrix}
				\mid & \mid & & \mid \\ 
				x_1 & x_2 & \ldots & x_n \\
				\mid & \mid & & \mid 
				\end{bmatrix} \nonumber \\
			&=\begin{bmatrix}\mid & \mid & & \mid \\
					\lambda_1 x_1 & \lambda_2 x_2 & \ldots & \lambda_n x_n \\
					\mid & \mid & & \mid \end{bmatrix} \nonumber \\
			&= \begin{bmatrix}\mid & \mid & & \mid \\
				x_1 & x_2 & \ldots & x_n \\
				       \mid & \mid & & \mid \end{bmatrix}
			  \begin{bmatrix} \lambda_1 & \ldots & 0 \\
					  	    & \ddots &   \\
						  	0   & \ldots & \lambda_n \end{bmatrix} \nonumber \\
						  	&= S \Lambda \nonumber \\ 	
						  	&\Rightarrow AS = S\Lambda
			\end{align*}
			
			\item $ A = S \Lambda S^{-1} \Rightarrow A^k = S \Lambda^{k} S^{-1}$
			
			\begin{Theorem}
			 $A^k \rightarrow 0 \text{ as } k \rightarrow \infty \text{ if all } |\lambda_i| < 1.$ 
			\end{Theorem}
			
			\item $A$ is sure to be diagonalizable if all the $\lambda_i$'s are different.
			
			\item  If we have repeated eigenvalues: may or may not have independent eigenvectors (consider $I$).
			
			\item Symmetric matrices: eigenvalues are real, eigenvectorss are orthogonal to each other

			\item $e^{At}$:
			\begin{align*}
				e^{At} &= I + At + \frac{{At}^2}{2} + \ldots + \frac{{At}^n}{n!} + \ldots \\
						&= SS^{-1} + S \Lambda S^{-1} t + S \Lambda^2 S^{-1} t/2+ \ldots \\
						&= S e^{\Lambda t} S^{-1}
			\end{align*}
			
			\item $e^{\Lambda t} = \begin{bmatrix}e^{\lambda_1 t} & \ldots & 0 \\
				& \ddots & \\
				0 & \ldots & e^{\lambda_n t}
			   \end{bmatrix}$

			\item \red{Note: assumed $A$ is diagonalizable.}
			
			\item eigenvalues of $A$ and $A^T$ are the same. $det(A-\lambda I) = det(A^T - \lambda I)$


		\end{itemize}
\end{itemize}
\end{document}
