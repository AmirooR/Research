\documentclass[a4paper,12pt]{article}
\usepackage{amsmath}
\usepackage{amssymb}
\usepackage{amsthm}
\usepackage{color}
\usepackage{mathtools}
\usepackage{tabularx}
\newcommand{\blue}[1]{\textcolor{blue}{#1}}
\newcommand{\red}[1]{\textcolor{red}{#1}}
\newcommand{\ud}{\,\mathrm{d}}
\theoremstyle{definition} \newtheorem{Theorem}{Theorem}
\begin{document}

\begin{center}
In The Name of God.

The Merciful, The Compassionate.
\vskip 1cm
{\Large\bfseries{Markov Matrices and Fourier Series}}

\vskip 0.2cm
\tiny{notes on Gilbert Strang videos, Lecture 24}
\end{center}

\section{Markov matrices}
\begin{enumerate}
\item all entries $\geq 0$
\item all columns add to $1$
\end{enumerate}
\begin{itemize}
	\item Steady state: $\rightarrow \lambda = 1$

	\item Two facts:
	\begin{enumerate}
		\item $\lambda = 1$ is an eigenvalue, $x_1 \geq 0$
		\begin{itemize}
			\item $A - 1I = $ matrix with all columns add to $0$. So, it is singular and
$1$ is an eigenvalue for $A$

			\item Also note that $\mathbf{1}^T\text{ is in }n((A-I)^T)$.
			
			\item $x_1$ is in $n(A-I)$
		\end{itemize}
		\item All other $|\lambda_i| < 1 $
	\end{enumerate}
	\item $\Rightarrow u_k = A^k u_0 = c_1 {\lambda_1}^k x_1 + c_2 {\lambda_2}^k x_2 + \ldots \qquad \rightarrow \qquad  c_1 x_1  \text{ as } k \rightarrow \infty$

	\item $x_1$ part of $u_0$ is the steady state.
\end{itemize}

\section{Projection with orthonormal basis $\{q_1, \ldots, q_n\}$}
\begin{itemize}
	\item Any $v = x_1 q_1 + x_2 q_2 + \ldots + x_n q_n $
	\item ${q_1}^T v = x_1, \ldots$
	\item $\begin{bmatrix}
		\mid & & \mid \\
		q_1&\ldots&q_n\\
		\mid & & \mid 
	   \end{bmatrix} \begin{bmatrix} x_1 \\ \vdots \\ x_n \end{bmatrix} = v$
	   
	\item $ Qx=v \Rightarrow x = Q^{-1}v =  Q^T v $
\end{itemize}

\section{Fourier series}
\begin{itemize}
	\item $f(x) = a_0 + a_1 cos(x) + b_1 sin(x) + a_2 cos(2x) + b_2 sin(2x) + \ldots $
	\item infinite, orthogonal bases \red{(working in function space)}.
	\item bases are $1, cos(x), sin(x) , cos(2x), sin(2x), \ldots $
	\item inner product of functions: $f^T g = \int f(x) g(x)\ud x =  \int_0^{2\pi} f(x)g(x) \ud x \rightarrow$ orthonormal infinite bases
	\item $a_1 = $?
	\item $\int_0^{2\pi} f(x)cos(x)dx = a_1 \int_0^{2\pi} (cos(x))^2\ud x $
	\item $\Rightarrow a_1 = \frac{1}{\pi} \int_0^{2\pi} f(x)cos(x)\ud x $
\end{itemize}
\end{document}
