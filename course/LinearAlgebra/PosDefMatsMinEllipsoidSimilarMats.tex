\documentclass[a4paper,12pt]{article}
\usepackage{amsmath}
\usepackage{amssymb}
\usepackage{amsthm}
\usepackage{color}
\usepackage{mathtools}
\usepackage{tabularx}
\newcommand{\blue}[1]{\textcolor{blue}{#1}}
\newcommand{\red}[1]{\textcolor{red}{#1}}
\newcommand{\ud}{\,\mathrm{d}}
\theoremstyle{definition} \newtheorem{Theorem}{Theorem}
\begin{document}

\begin{center}
In The Name of God.

The Merciful, The Compassionate.
\vskip 1cm
{\Large\bfseries{Positive Definite Matrices, Minima, Ellipsoids, and Similar Matrices }}

\vskip 0.2cm
\tiny{notes on Gilbert Strang videos, Lecture 27,28}
\end{center}

\section{Positive Definite Matrices}
\begin{itemize}
\item Positive Definiteness Test: $ A = \begin{bmatrix}
						a & b \\
						b & c
					 \end{bmatrix}$
\begin{enumerate}
\item Eigen values must be greater than zero $\lambda_1 > 0, \lambda_2 > 0$
\item Sub-determinants should be greater than zero: $a > 0, ac - b^2 > 0$
\item Pivots should be greater than zero: $a>0, \frac{ac-b^2}{a} > 0 $
\item Definition: $x^T A x > 0$ at all points except at $x=0$
\end{enumerate}
\item Semi-definite: all $>$ becomes $\geq$.
\item $x^T A x = ax^2 + 2bxy + cy^2, x = \begin{bmatrix}x \\ y\end{bmatrix}$
\item Cutting this equation at level one: gives an ellipse
\item MIN: 
\begin{itemize}
\item  Calculus: $\frac{\ud^2 u}{\ud x^2} > 0$ and $\frac{\ud u}{\ud x} = 0$
\item Linear Algebra: min of $f(x_1, x_2,\ldots,x_n)$: Matrix of second derivatives 
should be positive definite.
\end{itemize}
\item Matrix of second derivatives: $\begin{bmatrix} f_{xx} & f_{xy} \\ f_{yx} & f_{yy}\end{bmatrix}$
\item In higher dimensions, (e.g dim=3) : cutting at level one gives an ellipsoid.
\item $A = Q \Lambda Q^T$. So, the ellipoid has $n$ principal axes parallel to eigenvectors, and their scales
 are given by eigenvalues.
\item Some notes about positive definite (PD) matrices: 
\begin{itemize}
\item if $A$ is PD: $A^{-1}$ is also PD. (eigenvalues are positive)
\item if $A$ and $B$ are PD matrices: $A+B$ is also PD. ($x^T A x > 0\text{ and } x^T B x > 0$)
\item if $A$ is a m-by-n matrice: $A^T A$ is symmetric, square, and positive semi-definite! 
\begin{equation*}
x^T A^T A x = \lVert Ax \rVert^2 \geq 0
\end{equation*}

\item if $A$ has rank $n$ ( $n$ independent columns, or $N(A)=\{\}$ ), $A^T A$ becomes PD.
\end{itemize}
\end{itemize}

\section{Similar Matrices}
\begin{itemize}
\item Two n-by-n matrices $A$ and $B$ are \textbf{similar} means: for some invertible matrice $M$,
 $B = M^{-1} A M$.
\item Similar matrices have the same $\lambda$'s!!
		\begin{align*}
		Ax &= \lambda x \\
		AMM^{-1}x &= \lambda x \\
		M^{-1}AM M^{-1} x &= \lambda M^{-1} x \\
		B M^{-1}x &= \lambda M^{-1}x
		\end{align*}

\item Eigenvecotrs of $B$ is $M^{-1}(\text{eigenvectors of } A)$
\item \red{BAD CASE Example: $\lambda_1 = \lambda_2 = 4$}:
\begin{enumerate}
\item one small family: $\begin{bmatrix}4 & 0 \\ 0 & 4\end{bmatrix}$

			\begin{equation*}
			M^{-1}\begin{bmatrix}4 & 0 \\ 0 & 4\end{bmatrix}M = \begin{bmatrix}4 & 0 \\ 0 & 4\end{bmatrix}
			\end{equation*}

			It is only similar to itself!
\item big family includes $\begin{bmatrix}4 & 1 \\ 0 & 4\end{bmatrix} \leftarrow \text{ Jordan From}$

\item More members of family:$\begin{bmatrix}4 & 1 \\ 0 & 4\end{bmatrix}, \begin{bmatrix}5 & 1 \\ -1 & 3\end{bmatrix},
\begin{bmatrix}4 & 0 \\ 17 & 4\end{bmatrix} $

\end{enumerate}
\end{itemize}
\end{document}
