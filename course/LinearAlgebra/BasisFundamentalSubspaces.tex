\documentclass[a4paper,12pt]{article}
\usepackage{amsmath}
\usepackage{amssymb}
\usepackage{color}
\usepackage{tabularx}
\newcommand{\blue}[1]{\textcolor{blue}{#1}}
\newcommand{\red}[1]{\textcolor{red}{#1}}
\begin{document}

\begin{center}
In The Name of God.

The Merciful, The Compassionate.
\vskip 1cm
{\Large\bfseries{Basis, 4 Fundamental Subspaces}}

\vskip 0.2cm
\tiny{notes on Gilbert Strang videos, Lecture 09,10,11}
\end{center}

\section{Linear independence, Spanning a space, Basis, and Dimension}
\begin{itemize}
	\item independence is obvious!
	\item Vectors $\{v_1, \ldots v_n\}$ span a space means: The space consists all combinations of those vectors.
	\item \textbf{Basis}: Basis for a space is a sequence of vectors $\{v_1, \ldots, v_d\}$ that has two properties:
		\begin{enumerate}
			\item They are independent
			\item They span a space
		\end{enumerate}
	\item Given a space, every basis for the space has the same number of vectors: This number is the dimension of the space. (proof, see Carlo Tomasi's note)
	\item $Rank(A)= \# \text{pivot columns} = \text{dimension of the column space}$
	\item $dim(C(A)) = r, dim(N(A)) = \# \text{free variables} = n - r$
\end{itemize}

\section{The 4 fundamental subspaces}
\begin{itemize}
	\item 4 subspaces:
	\begin{itemize}
		\item $A \text{is a} m \times n$ Matrix
		\item $C(A),N(A), N(A^T) \text{ or left } N(A), R(A) = C(A^T)$
	\end{itemize}
	\item $\mathbf{\mathbb{R}}^n: R(A) \perp N(A)$ 
	\item $\mathbf{\mathbb{R}}^m: C(A) \perp N(A^T)$
	\item $C(rref(A)) \neq C(A)$
	\item $R(rref(A)) = R(A)$, row operations $\Longrightarrow dim(R(A)) = r$
	\item $y^T A = \mathbf{0}^T $, $y$ is in left nullspace of $A$,\\
	\\
		$rref(\begin{bmatrix}
		A_{m \times n} & & \mathcal{I} 
		\end{bmatrix}) \longrightarrow    
		\begin{bmatrix}
R_{m \times n} & & E_{m \times m}
		\end{bmatrix}		
 := E
 \begin{bmatrix}
 A& & I
 \end{bmatrix}
  = \begin{bmatrix}
   R& & E
   \end{bmatrix}$ \\
   $ \Longrightarrow EA = R$, \\
   \\
		 if $A$ was square and invertible: $R = I \Rightarrow E = A^{-1}$, rows of $E$ corresponding to zero rows of $R$ are basis for $N(A.T)$, because they are row vectors producing zero when multiplied by $A$.
		 
		 \item New vector spaces: $M = $ all $m\,by\,m$ matrices.
		 \item Subspaces of $M$: 
		 \begin{itemize}
		 	\item All upper triangular matrices ($U$)
		 	\item All symmetric matrices ($S$)
		 	\item Diagonal matrices: $D = S \cap U$: dim(D) = m
		 	\item $S+U$:any element in $S +$ any element in $U$: all $m\,by\,m$ matrices: dim=$m^2$
			\item $dim(S)+dim(U) - dim(S \cap U) = dim(S \cup U)$		 	
		 \end{itemize}
\end{itemize}
%\begin{center}
\begin{tabularx}{\linewidth}[t]{|c|c|c|X|X|}
		\hline
		& $C(A)$ & $N(A)$ & $R(A)$ & $N(A^T)$ \\
		\hline
		Basis: & Pivot Cols & Special Sol'n & First $r$ rows of $R:=rref(A)$ & rows of $E$ corresponding to zero rows of $R$\\
		\hline
		DIM: & $r$ & $n-r$ & $r$ & $m-r$ \\
		\hline
		\end{tabularx}		
%\end{center}
\end{document}