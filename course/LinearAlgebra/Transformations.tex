\documentclass[a4paper,12pt]{article}
\usepackage{amsmath}
\usepackage{amssymb}
\usepackage{amsthm}
\usepackage{color}
\usepackage{mathtools}
\usepackage{tabularx}
\newcommand{\blue}[1]{\textcolor{blue}{#1}}
\newcommand{\red}[1]{\textcolor{red}{#1}}
\newcommand{\ud}{\,\mathrm{d}}
\theoremstyle{definition} \newtheorem{Theorem}{Theorem}
\begin{document}

\begin{center}
In The Name of God.

The Merciful, The Compassionate.
\vskip 1cm
{\Large\bfseries{Linear Transformations, Change of basis}}

\vskip 0.2cm
\tiny{notes on Gilbert Strang videos, Lecture 30-32}
\end{center}

\section{Linear Transformations}
\begin{itemize}
\item $T(cV+dW) = cT(V) + dT(w)$
\item $T(0) = 0$
\item $T(x) = Ax$ is a linear transformation
\item Coordinates come from a basis: coordinates of $v = c_1 v_1 + c_2 v_2 + \cdots + c_n v_n$ are $c_1, c_2, \cdots $
\item Construct matrix $A$ that represents linear transformation $T$:
\begin{itemize}
\item $T: \mathbb{R}^n \rightarrow \mathbb{R}^m$
\item Choose bases $v_1, \cdots, v_n$ for inputs.
\item Choose bases $w_1, \cdots, w_m$ for outputs.
\end{itemize}
\item Rule to find $A$ given bases $v_1, \cdots, v_n$ and $w_1, \cdots, w_m$:
\begin{itemize}
\item  first column of $A$: write $T(v_1) = a_{11} w_1 + a_{21} w_2 + \cdots a_{m1} w_m$
\item second column of $A$: write $T(v_2) = a_{12} w_1 + \cdots a_{m2} w_m$
\item $A \left[ \text{input coordinates} \right] = \left[ \text{output coordinates} \right]$
\end{itemize}
\item Example: $T=\frac{\ud}{\ud x}$
\begin{itemize}
\item Input: $c_1 + c_2 x + c_3 x^2$, basis: $1, x, x^2$
\item Output: $c_2 + 2 c_3 x$, basis: $1, x$
\item $A\begin{bmatrix}c_1 \\ c_2 \\ c_3\end{bmatrix} = \begin{bmatrix} c_2 \\ 2 c_3\end{bmatrix}$
\item $A = \begin{bmatrix} 0 & 1 & 0  \\ 0 & 0 & 2\end{bmatrix}$
\end{itemize}
\end{itemize}

\section{Change of basis}
\begin{itemize}
\item Columns of $W =$ new basis vectors
\item $[x]_{\text{old basis}} \rightarrow [c]_{\text{new basis}}: x = W c$
\end{itemize}
\section{Transformation Matrices}
\begin{itemize}
\item Transformation matrices with respect to different bases:
\begin{itemize}
\item $T$ with respect to $v_1,\cdots, v_8$ it has matrix $A$
\item with respect to $w_1,\cdots, w_8$ it has matrix $B$
\item $A$ is similar to $B$, $A = M^{-1} B M$, $M$ is the change of basis vector
\end{itemize}
\item What is $A$? using $v_1, \ldots, v_8$.
\begin{itemize}
\item know $T$ completely from $T(v_1), T(v_2), \cdots, T(v_8)$.
\item Because every $x = c_1 v_1 + c_2 v_2 + \cdots, c_8 v_8$, Then $T(x) = c_1 T(v_1)  + \cdots$ .
\begin{itemize}
\item Write $T(v_1) = a_{11} v_1 + a_{21} v_2 + \cdots + a_{81} v_8$
\item $T(v_2) = a_{12} v_1 + a_{22} v_2 + \cdots + a_{82} v_8$
\end{itemize}
\end{itemize}
\item Eigenvector basis:
\begin{itemize}
\item $T(v_i) = \lambda_i v_i$
\item $\Rightarrow A = \begin{bmatrix}\lambda_1 & \cdots & 0 \\ & \ddots & \\ 0 & \cdots & \lambda_n \end{bmatrix}$
\end{itemize}
\end{itemize}
\textbf{A note:} When we have orthogonal eigenvectors: $A A^T = A^T A$ \\
symmetric, skew symmatric (i.e., $A^T = -A$), and orthogonal matrices pass this test!
\end{document}
