\documentclass[a4paper,12pt]{article}
\usepackage{amsmath}
\usepackage{color}
\newcommand{\blue}[1]{\textcolor{blue}{#1}}
\newcommand{\red}[1]{\textcolor{red}{#1}}
\begin{document}

\begin{center}
In The Name of God.

The Merciful, The Compassionate.
\vskip 1cm
{\Large\bfseries{Linear Equations}}

\vskip 0.2cm
\tiny{notes on Gilbert Strang videos, Lecture 07,08}
\end{center}

\section{Solutions to $Ax=b$}
\begin{itemize}
\item Rank $r$ is the number of pivots in elimination.
\item Number of free variables equals to the number of columns without pivot.
\item $R = [ I F;0 0 ]$, reduced echelon form.
\item $N$ is null space matrix (columns are special sol'n): $RN=0 \Rightarrow N = [ -F; I ]$
\item $Rx = 0 \Rightarrow [I F][x_{pivot}; x_{free}] = 0 \Rightarrow x_{pivot} = -Fx_{free}$
\item $r \leq M, r \leq N$
\item If $A$ is an $M \times N$ matrix, use the following rules to know about the solutions:
\begin{enumerate}
\item \blue{if $r = N < M$ (Full column rank matrix):}
\begin{itemize}
\item $N(A) = \{ \text{zero vector} \}$, because we have zero free variables.
\item $x_{complete} = x_{particular}$ if a solution exists.
\item $R = [ I; 0 ]$
\item \red{zero or one solution}
\end{itemize}
\item \blue{if $r = M < N$ (Full row rank matrix):}
\begin{itemize}
\item $N-M$ free variables.
\item $R=[IF]$
\item has solution for every b!
\item \red{infinite number of solutions}
\end{itemize}
\item \blue{if $r = M = N$:}
\begin{itemize}
\item $R = I$
\item invertible
\item \red{unique solution}
\end{itemize}
\item \blue{if $r < M, r < N$:}
\begin{itemize}
\item $R = [ I F; 0 0]$
\item if solution exists: $x_{complete} = x_{particular} + x_{nullspace}$
\item \red{zero or infinite solutions}
\end{itemize}
\end{enumerate}

\item In finding special solutions we have $r$ pivot columns and $n - r$ free variables.
we can set the free variables. e.g. to 1-hot encoding and find the other variables' values. Any combination of
special solutions are also a special solution (they are solutions of $Ax=0$)

\end{itemize}

\end{document}