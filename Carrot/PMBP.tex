\documentclass[a4paper,12pt]{article}
\usepackage{amsmath}
\usepackage{amssymb}
\begin{document}

\begin{center}
In The Name of God.

The Merciful, The Compassionate.
\vskip 1cm
{\Large\bfseries{PMBP: PatchMatch Belief Propagation for

Correspondence Field Estimation}}

\vskip 0.2cm
\tiny{By Fredric Besse, Carsten Rother, Andrew Fitzgibbon, Jan Kautz}
\end{center}

\section{Abstract and Introduction}
This paper draws a new connection between PatchMatch(PM) and Particle Belief Propagation (PBP).
The key contributions are as follows:
\begin{enumerate}
\item description of PM and BP in terms that allow the connection between them be clearly described.
\item use of this analysis to define a new algorithm PMBP which is more accurate than PM and faster than BP. 
\end{enumerate}

\subsection*{Belief Propagation (BP)}
Correspondence field is parametrized by a vector grid $\{\mathbf{u}_s\}_{s=1}^{n}$ where $s$ indexes nodes ( correspondence to image pixels) and  $\mathbf{u}_s \in \mathbb{R}^d$ parametrizes the correspondence vector at $s$.

Problems with the data term for weighted patch flow are as follows:
\begin{enumerate}
\item it implicitly assumes a constant correspondence field in the $(2h+1)\times(2h+1)$ patch surrounding every pixel (?). Large $h$ over-smooths the output. More complex parametrization of flow field can be suggested. However, they are not computationally tractable.
\item $h$ may be reduced. This causes the data term to be ambiguous. This causes the introduction of pairwise terms.
\end{enumerate}
For discrete problems, where $\mathbf{u}$ live in finite set of size $D$, energy can be minimized in $O(nD)$ time, while with pairwise terms, the worst case complexity is $O(D^n)$. BP finds good minimizers in times far below the worst case. However, for correspondence problems, where $\mathbf{u}$ lives in an effective continuous space, $D$ must be very large. So, even minimizing unary only costs is too expensive.

\subsection*{PatchMatch}
An efficient way to compute a nearest neighbour field (NNF) between 2 images. It is considered as an efficient global minimizer for unary term only energies.

In PatchMatch Stereo [3] disparity is over-parametrized by 3D vector $\mathbf{u}_s = [ a_s, b_s, c_s ]^T$, parametrizing planar disparity surface $\Delta_s(x,y)=a_s(x-x_s)+b_s(y-y_s)+c_s$. 

A key deficiency of PM is that it lack an explicit smoothness control on output field (has deficiency in finding reliable correspondences in very large smooth regions).
A related deficiency is tendency of PM to require a form of early stopping: global optimum of unary energy is not necessarily the best solution in terms of image error (?) $\rightarrow$ \textit{error versus energy trade-off}. 
\begin{align*}
f(S) &:= \{f(s)|s \in S\} \\
\text{fargmin}_K(S,f) &:= S_K \subset S \, \text{s.t.} \, |S_K| = \min(K,|S|) \,\text{and}\, \max f(S_K) \leq \min f(S \backslash S_K)
\end{align*}

\subsection{PatchMatch with Particles}
A set of $K$ particles $P_s \subset \mathbb{R}^d$ is associated to each node $s$. Each particle $p \in P_s$ is a candidate solution $\mathbf{u}_s^{*}$ for the minimization problem. These sets are initialized randomly (data driven initialization is also applicable).

One PM iteration is linear sweep through all nodes. The order is defined by $\phi(s)$ so that $s$ is visited before $s'$ if $\phi(s) < \phi(s')$. Predecessor set $\Phi_s = \lbrace s'|\phi(s')<\phi(s) \rbrace$. Usually the order is from top-left to bottom-right and inverse on odd/even iterations.

Two update steps are performed:
\begin{enumerate}
\item Propagation: $P_s$ is updated: 
\begin{align*}
C_s &= \bigcup \lbrace P_t | t \in N(s) \cap \Phi_s \rbrace,\\ 
P_s &\leftarrow \text{fargmin}_K (P_s \cup C_s, \psi_s)
\end{align*}
%$$
\item Resampling: perturbs particles locally by $\mathcal{N}(0,\sigma)$:
\begin{align*}
R_s &= \lbrace p + \mathcal{N}(0,\sigma) | p \in P_s \rbrace, \\
P_s &\leftarrow \text{fargmin}_K (P_s \cup R_s, \psi_s)
\end{align*}
\end{enumerate}

\subsection{Particle Belief Propagation}

\end{document}