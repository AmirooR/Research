\documentclass[a4paper,12pt]{article}
\usepackage{amsmath}
\usepackage{amssymb}
\begin{document}

\begin{center}
In The Name of God.

The Merciful, The Compassionate.
\vskip 1cm
{\Large\bfseries{PMBP: PatchMatch Belief Propagation for

Correspondence Field Estimation}}

\vskip 0.2cm
\tiny{By Fredric Besse, Carsten Rother, Andrew Fitzgibbon, Jan Kautz}
\end{center}

\section{Abstract and Introduction}
This paper draws a new connection between PatchMatch(PM) and Particle Belief Propagation (PBP).
The key contributions are as follows:
\begin{enumerate}
\item description of PM and BP in terms that allow the connection between them be clearly described.
\item use of this analysis to define a new algorithm PMBP which is more accurate than PM and faster than BP. 
\end{enumerate}

\subsection{Belief Propagation}
Correspondence field is parametrized by a vector grid $\{\mathbf{u}_s\}_{s=1}^{n}$ where $s$ indexes nodes ( correspondence to image pixels) and  $\mathbf{u}_s \in \mathbb{R}^d$ parametrizes the correspondence vector at $s$.

Problems with the data term for weighted patch flow are as follows:
\begin{enumerate}
\item it implicitly assumes a constant correspondence field in the $(2h+1)\times(2h+1)$ patch surrounding every pixel (?). Large $h$ over-smooths the output. More complex parametrization of flow field can be suggested. However, they are not computationally tractable.
\item $h$ may be reduced. This causes the data term to be ambiguous. This causes the introduction of pairwise terms.
\end{enumerate}
\end{document}