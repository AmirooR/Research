\documentclass[a4paper,12pt]{article}
\usepackage{amsmath}
\begin{document}

\begin{center}
In The Name of God.

The Merciful, The Compassionate.
\vskip 1cm
{\Large\bfseries{Extracting 3D Scene-consistent Object

Proposals and Depth from Stereo Images}}

\vskip 0.2cm
\tiny{By Michael Bleyer, Christoph Rhemann, and Carsten Rother}
\end{center}

\section{Abstract and Introduction}
\begin{itemize}
\item The goal is to jointly extract objects and estimate depths from stereo images
\item Main contribution is to introduce the concept of 3D scene consistency in stereo matching
\item Few works on 3D reasoning with respect to stereo images
\item Object stereo [1]: the goal was to improve depth estimation by object extraction.
\item This work: main focus is on object extraction.
\item Inspired by the work of [12]. Proposed the following 3-step pipeline for object extraction:
\begin{enumerate}
\item generate large pool of object proposals
\item rank object proposals by learning objectness score
\item perform object recognition on top ranked proposals
\end{enumerate}
\item This work differs in the case that it takes an stereo image as input and generates a pool of scene proposals which consist:
\begin{enumerate}
\item disparity map
\item object map: each pixel $\longrightarrow$ an object
\end{enumerate}
\item Object stereo [1]: did not introduce the concept of computing a pool of object maps.
\item Key difference is objects in [1] were approximated by flat 2D planes. We enclose them by using a 3D bounding box
$\Longrightarrow$ we can exploit physical constraints.
\end{itemize}

\section{Model}
Each pixel $p \in \mathcal{I}$ is assigned to a 3D plane. Computes a mapping $F: \mathcal{I} \rightarrow \mathcal{F}$ where $\mathcal{F}$ denotes the set of all possible 3D planes. Disparity $d_p$ is defined using its plane $f_p$ as $d_p := a_{f_p}p_x+b_{f_p}p_y + c_{f_p}$. 

Second mapping for objects: $O: \mathcal{I} \rightarrow \mathcal{O}$ where $O$ is object map and $\mathcal{O}$ is 
set of all objects. An object is defined by 2 parameters:
\begin{enumerate}
\item Oriented 3D bounding box
\item Color model
\end{enumerate}
$\langle F,O\rangle$ forms the scene proposal.

The quality of a scene proposal is measured by an energy, as:
\begin{equation}
E(F,O) = E_{pc}(F) + E_{col}(O) + E_{ol}(O,F) +\\
E_{tight}(O) + E_{is}(O) + E_{gravity}(O)+E_{mdl}(O)
\end{equation}
The individual terms are explained informally next:
\begin{itemize}
\item $E_{pc}(F)$: photo consistency; penalizes difference between left and right image given $f_p$ with local smoothing.
\item $E_{col}(O)$: color; prefers objects that are compact in color. Color of an object is modelled by GMM.
\item $E_{ol}(O,F)$: Bounding box(BB) outlier; penalizes count of 3D points $P$ outside object $O_p$'s BB. (How BB is computed?)
\item $E_{tight}(O)$: tightness $\rightarrow$ $\sum_{o \in O} volume(o)$; penalizes BBs from unnecessarily fill free space.
\item $E_{is}(O)$: intersection.
\item $E_{gravity}(O)$: gravity; encourages objects to stand on top of each other
\item $E_{mdl}(O)$: mdl; encourages small number of objects as possible.
\end{itemize}

\section{Optimization}
\end{document}