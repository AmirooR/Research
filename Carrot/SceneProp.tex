\documentclass[a4paper,12pt]{article}
\usepackage{amsmath}
\begin{document}

\begin{center}
In The Name of God.

The Merciful, The Compassionate.
\vskip 1cm
{\Large\bfseries{Extracting 3D Scene-consistent Object

Proposals and Depth from Stereo Images}}

\vskip 0.2cm
\tiny{By Michael Bleyer, Christoph Rhemann, and Carsten Rother}
\end{center}

\section{Abstract and Introduction}
\begin{itemize}
\item The goal is to jointly extract objects and estimate depths from stereo images
\item Main contribution is to introduce the concept of 3D scene consistency in stereo matching
\item Few works on 3D reasoning with respect to stereo images
\item Object stereo [1]: the goal was to improve depth estimation by object extraction.
\item This work: main focus is on object extraction.
\item Inspired by the work of [12]. Proposed the following 3-step pipeline for object extraction:
\begin{enumerate}
\item generate large pool of object proposals
\item rank object proposals by learning objectness score
\item perform object recognition on top ranked proposals
\end{enumerate}
\item This work differs in the case that it takes an stereo image as input and generates a pool of scene proposals which consist:
\begin{enumerate}
\item disparity map
\item object map: each pixel $\longrightarrow$ an object
\end{enumerate}
\item Object stereo [1]: did not introduce the concept of computing a pool of object maps.
\item Key difference is objects in [1] were approximated by flat 2D planes. We enclose them by using a 3D bounding box
$\Longrightarrow$ we can exploit physical constraints.
\end{itemize}

\section{Model}
\begin{itemize}
\item 
\end{itemize}

%\subsection*{a)}
%The equations for the moving car are\\
%\\
%$x(t) = x(t-1) + \dot{x}(t-1) \times \Delta{t} + \frac{1}{2} \ddot{x}(t-1) \Delta{t}^2$\\
%$\Delta{t} = 1$\\
%$x(t) = x(t-1) + \dot{x}(t-1) + \frac{1}{2} \ddot{x}(t-1)$\\
%$\ddot{x} \sim N(0,1)$\\
%$x(t) = x(t-1) + \dot{x}(t-1) + \frac{1}{2} N(0,1)$\\
%$\dot{x}(t) = \dot{x}(t-1) + \ddot{x}(t-1) \Delta{t} = \dot{x}(t-1) + \ddot{x}(t-1)$\\
%
%The position x(t) depends on the previous velocity and position at time t-1.
%
%We can define the state as
%\begin{equation*}
%	\mathbf{X_t} = \left(
%		\begin{array}{c}
%		x(t) \\
%		\dot{x}(t)
%		\end{array} \right)
%\end{equation*}
%
%\subsection*{b)}
%
%The state transition probability $p(x_{t} | u_{t}, x_{t-1})$ must be linear function in its arguments with added gaussian noise:
%\begin{equation*}
%	\mathbf{X_{t} = A_{t} X_{t-1} + B_{t} u_{t} + \epsilon_{t} } 
%\end{equation*}
%
%so,
%
%\begin{equation*}
%	\mathbf{X_t} = \left(
%		\begin{array}{c}
%		x(t) \\
%		\dot{x}(t)
%		\end{array} \right) = 
%		\overbrace{
%		\left(
%			\begin{array}{cc}
%			1 & 1 \\
%			0 & 1 
%			\end{array} \right)}^{\mathbf{A_t} }
%			\left( 
%			\begin{array}{c} 
%				x(t-1) \\
%				\dot{x}(t-1)
%			\end{array} \right)
%			+
%			\overbrace{
%			\left( 
%			\begin{array}{c} 
%				0.5 \\
%				1
%			\end{array} \right)}^{\mathbf{B_t} }
%			N(0,1)
%\end{equation*}
\end{document}